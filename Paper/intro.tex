\documentclass[9pt]{article}
\usepackage{mathtools}
\usepackage{graphicx}
\usepackage{layout}
\setlength{\voffset}{-0.75in}
\setlength{\headsep}{5pt}
\begin{document}

\title{Dynamics Of A Quadcopter}
\author{Daniel Wood}
\maketitle
\section{Introduction}

In recent years the development of smaller and cheaper electronics, specifically inertial measurement systems, has led to a rise in popularity of lightweight unmanned aerial vehicles. Unmanned aerial vehicles come in many configurations, this paper focuses on a symmetric quadrotor configuration. With a rise in general usage of quadrotor vehicles, there has surfaced the need to autonomously control these vehicles, some applications include automated delivery services, military usage, aerial imagery. For these applications it is critical to acheive stable flight

The basis of autonomous control is a valid dynamics model of the vehicle, and a fast, robust controller. The dynamics of a quadrotor is not trivial, in general it should be known how the system is behaving from a fixed location. This leads to defining the dynamics of the quadrotor in a body frame, then rotate it to the global or fixed frame, this is a well known technique in 3D dynamics\cite{FanniMohamed2017AN6Q},\cite{FarameeVeeravat2014EotS},\cite{HaomiaoHuang2009Aaco},\cite{DenisKotarski2016CDFU}.

There are many researchers dedicating time to improving controller robustness through different control schemes such as Multiple Input Multiple Output state variable (MIMO)\cite{FarameeVeeravat2014EotS}, Linear Quadratic Regulator (LQR), Model Predictive Control (MPC), Sliding Mode Control (SMC). In general this research focuses on nuances of the choosen algoritm with the goal of improvment in speed as well as disturbance rejection\cite{DenisKotarski2016CDFU}. Because of the scope of the research, the dynamics model of the quadrotor is generally linearized for simplicity, or assumptions that were made lead to reduced order of the system\cite{FarameeVeeravat2014EotS}. Varying linearization methods have been used such as Feedback Linearization \cite{KhalifaA2012Maco}, which takes a non-linear system and utilizes a technique to transform it into a linear control system.
\bibliography{References}
\bibliographystyle{plain}
\end{document}