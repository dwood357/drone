\documentclass[9pt]{article}
\usepackage{mathtools}
\usepackage{graphicx}
\usepackage{layout}
\setlength{\voffset}{-0.75in}
\setlength{\headsep}{5pt}
\begin{document}

\title{Dynamics Of A Quadcopter}
\author{Daniel Wood}
\maketitle
\section{Introduction}

In recent years the development of smaller and cheaper electronics, specifically inertial measurement systems, has led to a rise in popularity of lightweight unmanned aerial vehicles. Unmanned aerial vehicles come in many configurations, this paper focuses on a symmetric quadrotor configuration. With a rise in general usage of quadrotor vehicles, there has surfaced the need to autonomously control these vehicles, some applications include automated delivery services, military usage, aerial imagery. The basis of autonomous control is a valid dynamics model of the vehicle, and a fast, robust controller.

There are many researchers dedicating time to improving autonmous control through different control algorithms such as PID, LQR, sliding mode control. In general this research focuses on nuances of the choosen algoritm with the goal of improvment in speed as well as disturbance rejection\cite{DenisKotarski2016CDFU}. Because of the scope of the research, the dynamics model of the quadrotor is generally linearized for simplicity, or assumptions that were made lead to reduced order of the system\cite{FarameeVeeravat2014EotS}.
\bibliography{References}
\bibliographystyle{plain}
\end{document}