\documentclass[10pt, letterpaper]{article}
\usepackage{mathtools}
\usepackage[margin=0.5in]{geometry}
\setcounter{MaxMatrixCols}{12}
\begin{document}
\title{Dynamics Of A Quadcopter}
\author{Daniel Wood}
\maketitle

\section{Generalized Coordinates}
In order to properly define the dynamics of a quadcopter, two coordinate systems will be defined, body \& inertial coordinate systems. In the inertial frame the position will be defined by $X_{G}$, $Y_{G}$, $Z_{G}$ with angles of rotation in the body frame defined as $\psi$,$\theta$,$\phi$, which allows to define the generalized coordinates q:

\[
q=\left[\begin{array}{cccccc}
X_{G} & Y_{G} & Z_{G} & \psi & \theta & \phi\end{array}\right]
\]

The location of the body frame defined from the origin of the inertial
frame is defined by the following:

\[
\overrightarrow{r_{G}}=X_{G}\overrightarrow{I}+Y_{G}\overrightarrow{J}+Z_{G}\overrightarrow{K}
\]

In the case of the quadcopter where there are four motors, there are
four forces produced by the motors defined in the positive $\overrightarrow{k}$direction
of the body frame, in general the forces are defined as:

\[
\overrightarrow{F_{i}}=\left[\begin{array}{c}
0\\
0\\
F_{i}
\end{array}\right]i=1:4
\]

Which will yeild a resultant force in the body frame of:

\[
\overrightarrow{F_{R}}=\sum_{i=1}^{4}\overrightarrow{F_{i}}-mg
\]

These forces will also need to be defined in the inertial frame, this
is acheived by the rotation matrix:

\[
\overrightarrow{F_{R}}=R(\psi,\theta,\phi)\cdot \overrightarrow{F_{R}}
\]

Where the rotation matrix is defined as:

\[
R(\psi,\theta,\phi)=\left[\begin{array}{ccc}
cos\psi cos\theta & cos\theta sin\psi & -sin\theta\\
cos\psi sin\phi sin\theta-cos\phi sin\psi & cos\phi cos\psi+sin\phi sin\psi sin\theta & cos\theta sin\phi\\
sin\phi sin\psi+cos\phi cos\psi sin\theta & cos\phi sin\psi sin\theta-cos\psi sin\phi & cos\phi cos\theta
\end{array}\right]
\]

Which can be expressed in terms of the generalized coordinates q as:

\[
R(q)=\left[\begin{array}{ccc}
\cos(q_{4})\cos(q_{5}) & \cos(q_{5})\sin(q_{4}) & \-sin(q_{5})\\
\cos(q_{4})\sin(q_{6})\sin(q_{5})\-cos(q_{6})\sin(q_{4}) & \cos(q_{6})\cos(q_{4})+\sin(q_{6})\sin(q_{4})\sin(q_{5}) & \cos(q_{5})\sin(q_{6})\\
\sin(q_{6})\sin(q_{4})+\cos(q_{6})\cos(q_{4})\sin(q_{5}) & \cos(q_{6})\sin(q_{4})\sin(q_{5})-\cos(q_{4})\sin(q_{6}) & \cos(q_{6})\cos(q_{5})
\end{array}\right]\]

which yields the force resultant in the inertial frame as:

\[
\overrightarrow{F_{R}}(q)=\left[\begin{array}{c}
-sin(q_{5})\cdot(F_{1}+F_{2}+F_{3}+F_{4}-mg)\\
cos(q_{5})\sin(q_{6})\cdot(F_{1}+F_{2}+F_{3}+F_{4}-mg)\\
cos(q_{6})\cos(q_{5})\cdot(F_{1}+F_{2}+F_{3}+F_{4}-mg)
\end{array}\right]
\]

The position of the forces in the body frame are defined by the position
vectors $\overrightarrow{r_{i}}$ (i=1:4) as follows:
\[
\overrightarrow{r_{1}}=\frac{a}{2}\overrightarrow{i}+\frac{b}{2}\overrightarrow{j}+0\overrightarrow{k}
\]

\[
\overrightarrow{r_{2}}=-\frac{a}{2}\overrightarrow{i}+\frac{b}{2}\overrightarrow{j}+0\overrightarrow{k}
\]

\[
\overrightarrow{r_{3}}=\frac{a}{2}\overrightarrow{i}-\frac{b}{2}\overrightarrow{j}+0\overrightarrow{k}
\]

\[
\overrightarrow{r_{4}}=-\frac{a}{2}\overrightarrow{i}-\frac{b}{2}\overrightarrow{j}+0\overrightarrow{k}
\]

Using the forces and their position vectors the moment about origin
G in the body frame is defined as:

\[
\overrightarrow{M_{G}}=\sum_{i=1}^{4}(\overrightarrow{r_{i}}x\overrightarrow{F_{i}})
\]

this will also need to be rotated into the inertial frame yielding:

\[
\overrightarrow{M_{G}}(q)=\left[\begin{array}{c}
cosq_{4}cosq_{5}M_{1}+cosq_{5}sinq_{4}M_{2}\\
-(cosq_{6}sinq_{4}-cosq_{4}sinq_{6}sinq_{5})M_{1}+(cosq_{6}cosq_{4}+sinq_{6}sinq_{4}sinq_{5})M_{2}\\
(sinq_{6}sinq_{4}+cosq_{6}cosq_{4}sinq_{5})M_{1}-(cosq_{4}sinq_{6}-cosq_{6}sinq_{4}sinq_{5})M_{2}
\end{array}\right]
\]

\[
M_{1}=\frac{b}{2}F_{1}+\frac{b}{2}F_{2}-\frac{b}{2}F_{3}-\frac{b}{2}F_{4}
\]

\[
M_{2}=\frac{a}{2}F_{1}-\frac{a}{2}F_{2}+\frac{a}{2}F_{3}-\frac{a}{2}F_{4}
\]

For a full depiction of the dynamics of the system in the global frame
first we start with looking at the local angular velocities as expressed
in the body frame:

\[
\omega_{body}=\left[\begin{array}{c}
\omega_{x}\\
\omega_{y}\\
\omega_{z}
\end{array}\right]=\left[\begin{array}{ccc}
-sin\theta & 0 & 1\\
cos\theta cos\phi & cos\phi & 0\\
cos\phi cos\theta & -sin\phi & 0
\end{array}\right]\left[\begin{array}{c}
\dot{\psi}\\
\dot{\theta}\\
\dot{\phi}
\end{array}\right]
\]

Expressing these in the generalized coordinates of q yields:

\[
\omega(q)=R(q)\cdot\omega_{body}
\]

With the local angular velocities and the moments in the body frame
rotated into the inertial, the generalized forces $Q_{i}$$(q)$ can
be defined by:

\[
Q_{j}=\overrightarrow{F}_{R}\frac{\partial\overrightarrow{r_{G}}}{\partial q_{j}}+\overrightarrow{M}_{G}\frac{\partial\overrightarrow{\omega}}{\partial\dot{q_{j}}}
\]

This allows for a full generalized force representation as:

\[
Q_{1}=\overrightarrow{F_{R}}\frac{\partial\overrightarrow{r_{G}}}{\partial q_{1}}
\]

\[
Q_{2}=\overrightarrow{F_{R}}\frac{\partial\overrightarrow{r_{G}}}{\partial q_{2}}
\]

\[
Q_{3}=\overrightarrow{F_{R}}\frac{\partial\overrightarrow{r_{G}}}{\partial q_{3}}
\]

\[
Q_{4}=\overrightarrow{F_{R}}\frac{\partial\overrightarrow{r_{G}}}{\partial q_{4}}+\overrightarrow{M_{G}}\frac{\partial\omega}{\partial\dot{q_{4}}}
\]

\[
Q_{5}=\overrightarrow{F_{R}}\frac{\partial\overrightarrow{r_{G}}}{\partial q_{5}}+\overrightarrow{M_{G}}\frac{\partial\omega}{\partial\dot{q_{5}}}
\]

\[
Q_{6}=\overrightarrow{F_{R}}\frac{\partial\overrightarrow{r_{G}}}{\partial q_{6}}+\overrightarrow{M_{G}}\frac{\partial\omega}{\partial\dot{q_{6}}}
\]


\subsection*{The Kinetic Energy and the Equations of Motion}

The kinetic energy of the quadcopter is defined as:

\[
T(q)=\frac{1}{2}\dot{q}^{T}M(q)*\dot{q}
\]

where M(q) is defined as:

\[
M(q)=J_{v}(q)^{T}m_{rr}J_{v}(q)+J_{w}(q)^{T}m_{\theta\theta}J_{w}(q)
\]

with:

\[
J_{v}(q)=\left[\begin{array}{cccccc}
\frac{\partial q_{1}}{\partial q_{1}} & \frac{\partial q_{1}}{\partial q_{2}} & \frac{\partial q_{1}}{\partial q_{3}} & 0 & 0 & 0\\
\frac{\partial q_{2}}{\partial q_{1}} & \frac{\partial q_{2}}{\partial q_{2}} & \frac{\partial q_{2}}{\partial q_{3}} & 0 & 0 & 0\\
\frac{\partial q_{3}}{\partial q_{1}} & \frac{\partial q_{3}}{\partial q_{1}} & \frac{\partial q_{3}}{\partial q_{1}} & 0 & 0 & 0
\end{array}\right]
\]

\[
J_{w}(q)=\left[\begin{array}{cccccc}
0 & 0 & 0 & -sinq_{5} & 0 & 1\\
0 & 0 & 0 & cosq_{5}sinq_{6} & cosq_{6} & 0\\
0 & 0 & 0 & cosq_{5}cosq_{6} & -sinq_{6} & 0
\end{array}\right]
\]

\[
m_{rr}=\left[\begin{array}{ccc}
m & 0 & 0\\
0 & m & 0\\
0 & 0 & m
\end{array}\right]
\]

\[
m_{\theta\theta}=\left[\begin{array}{ccc}
I_{xx} & 0 & 0\\
0 & I_{yy} & 0\\
0 & 0 & I_{zz}
\end{array}\right]
\]

which expands to:

\[
T(q)=\frac{1}{2}[m\dot{q_{1}}^{2}+m\dot{q_{2}}^{2}+m\dot{q_{3}}^{2}+\dot{q_{4}}^{2}(I_{zz}cos(q_{5})^{2}cos(q_{6})^{2}+I_{yy}cos(q_{5})^{2}sin(q_{6})^{2}+I_{xx}sin(q_{5})^{2})+\ldots
\]

\[
\ldots+\dot{q_{5}}^{2}(I_{yy}cos(q_{6})^{2}+I_{zz}sin(q_{6})^{2})+2\dot{q_{5}}\dot{q_{4}}\sigma+\dot{q_{6}}^{2}I_{xx}-2\dot{q_{6}}\dot{q_{4}}I_{xx}sin(q_{5})]
\]

where:

\[
\sigma=I_{yy}cos(q_{5})cos(q_{6})sin(q_{6})-I_{zz}cos(q_{5})cos(q_{6})sin(q_{6})
\]

The potential energy of the drone is defined as:

\[
\overrightarrow{V}(q)=mgq_{3}
\]

which allows for the lagrangian to be:

\[
\mathcal{L}(q)=T(q)-V(q)
\]

with $\mathcal{L}(q)$ defined this allows us to find the corresponding
Euler-Lagrange equations of motions from the following equation:

\[
\frac{d}{dt}(\frac{\partial\mathcal{L}(q)}{\partial\dot{q_{j}}})-\frac{\partial\mathcal{L}(q)}{\partial q_{j}}=Q_{j}(q)
\]

\[
% m \ddot{q}_{1}
Q_{1}(q)=m\ddot{q_{1}}
\]

\[
Q_{2}(q)=m\ddot{q_{2}}
\]

\[
Q_{3}(q)=m\ddot{q_{3}}-mg
\]

\[
Q_{4}(q) = 2 I_{xx} {sin}(q_{5}) {cos}(q_{5}) \dot{q}_{4} \dot{q}_{5} -  I_{xx} {sin}(q_{5}) \ddot{q}_{6} -  I_{xx} {cos}^{2}(q_{5}) \ddot{q}_{4} -  I_{xx} {cos}(q_{5}) \dot{q}_{5} \dot{q}_{6}\ldots
\]

\[
\ldots +  I_{xx} \ddot{q}_{4} + \frac{1}{4} I_{yy} (- (\dot{q}_{5} - 2 \dot{q}_{6}) {cos}(q_{5} - 2 q_{6}) + (\dot{q}_{5} + 2 \dot{q}_{6}) {cos}(q_{5} + 2 q_{6})) \dot{q}_{5} + \frac{1}{4} I_{yy} (- {sin}(q_{5} - 2 q_{6}) + {sin}(q_{5} + 2 q_{6})) \ddot{q}_{5}\ldots
\]

\[
\ldots + 2 I_{yy} {sin}(q_{5}) {cos}(q_{5}) {cos}^{2}(q_{6}) \dot{q}_{4} \dot{q}_{5} - 2 I_{yy} {sin}(q_{5}) {cos}(q_{5}) \dot{q}_{4} \dot{q}_{5} + 2 I_{yy} {sin}(q_{6}) {cos}^{2}(q_{5}) {cos}(q_{6}) \dot{q}_{4} \dot{q}_{6}\ldots
\]

\[\ldots -  I_{yy} {cos}^{2}(q_{5}) {cos}^{2}(q_{6}) \ddot{q}_{4} +  I_{yy} {cos}^{2}(q_{5}) \ddot{q}_{4} + \frac{1}{4} I_{zz} (- (\dot{q}_{5} - 2 \dot{q}_{6}) {cos}(q_{5} - 2 q_{6}) + (\dot{q}_{5} + 2 \dot{q}_{6}) {cos}(q_{5} + 2 q_{6})) \dot{q}_{5}\ldots
\]

\[\ldots + \frac{1}{4} I_{zz} (- {sin}(q_{5} - 2 q_{6}) + {sin}(q_{5} + 2 q_{6})) \ddot{q}_{5} - 2 I_{zz} {sin}(q_{5}) {cos}(q_{5}) {cos}^{2}(q_{6}) \dot{q}_{4} \dot{q}_{5}\ldots
\]

\[\ldots - 2 I_{zz} {sin}(q_{6}) {cos}^{2}(q_{5}) {cos}(q_{6}) \dot{q}_{4} \dot{q}_{6} +  I_{zz} {cos}^{2}(q_{5}) {cos}^{2}(q_{6}) \ddot{q}_{4}
\]

\[
Q_{5}(q) = -  I_{xx} {sin}(q_{5}) {cos}(q_{5}) \dot{q}_{4}^{2} +  I_{xx} {cos}(q_{5}) \dot{q}_{4} \dot{q}_{6} + \frac {1}{4} I_{yy} (- (\dot{q}_{5} - 2 \dot{q}_{6}) {cos}(q_{5} - 2 q_{6}) + (\dot{q}_{5} + 2 \dot{q}_{6}) {cos}(q_{5} + 2 q_{6})) \dot{q}_{4}\ldots
\]

\[
\ldots + \frac{1}{4} I_{yy} (- {sin}(q_{5} - 2 q_{6}) + {sin}(q_{5} + 2 q_{6})) \ddot{q}_{4} - \frac{1}{4} I_{yy} (- {cos}(q_{5} - 2 q_{6}) + {cos}(q_{5} + 2 q_{6})) \dot{q}_{4} \dot{q}_{5}\ldots
\]

\[
\ldots-  I_{yy} {sin}(q_{5}) {cos}(q_{5}) {cos}^{2}(q_{6}) \dot{q}_{4}^{2} +  I_{yy} {sin}(q_{5}) {cos}(q_{5}) \dot{q}_{4}^{2}- 2.0 I_{yy} {sin}(q_{6}) {cos}(q_{6}) \dot{q}_{5} \dot{q}_{6}  \ldots
\]

\[
\ldots+  I_{yy} {cos}^{2}(q_{6}) \ddot{q}_{5} + \frac{1}{4} I_{zz} (- (\dot{q}_{5} - 2 \dot{q}_{6}) {cos}(q_{5} - 2 q_{6}) + (\dot{q}_{5} + 2 \dot{q}_{6}) {cos}(q_{5} + 2 q_{6})) \dot{q}_{4}\ldots
\]

\[
\ldots + \frac{1}{4} I_{zz} (- {sin}(q_{5} - 2 q_{6}) + {sin}(q_{5} + 2 q_{6})) \ddot{q}_{4} - \frac{1}{4} I_{zz} (- {cos}(q_{5} - 2 q_{6}) + {cos}(q_{5} + 2 q_{6})) \dot{q}_{4} \dot{q}_{5} \ldots
\]

\[\ldots +  I_{zz} {sin}(q_{5}) {cos}(q_{5}) {cos}^{2}(q_{6}) \dot{q}_{4}^{2} + 2.0 I_{zz} {sin}(q_{6}) {cos}(q_{6}) \dot{q}_{5} \dot{q}_{6} -  I_{zz} {cos}^{2}(q_{6}) \ddot{q}_{5} +  I_{zz} \ddot{q}_{5}
\]

\[
Q_{6}(q)=-I_{xx}sin(q_{5})\ddot{q}_{4} - I_{xx}cos(q_{5})\dot{q}_{4}\dot{q}_{5} + I_{xx}\ddot{q}_{6} - \frac{1}{4}I_{yy}(2cos(q_{5} - 2 q_{6}) + 2cos(q_{5} + 2 q_{6}))\dot{q}_{4}\dot{q}_{5}\ldots
\]

\[
\ldots - I_{yy}sin(q_{6})cos(q_{5})^{2}cos(q_{6})\dot{q}_{4}^{2} + I_{yy} sin(q_{6})cos(q_{6})\dot{q}_{5}^{2} - \frac{1}{4} I_{zz} (2cos(q_{5} - 2 q_{6}) + 2cos(q_{5} + 2 q_{6})) \dot{q}_{4}\dot{q}_{5}\ldots
\]

\[
\ldots + I_{zz}sin(q_{6})cos(q_{5})^{2}cos(q_{6})\dot{q}_{4}^{2} - I_{zz} sin(q_{6})cos(q_{6})\dot{q}_{5}^{2}
\]

\subsection*{State Space Representation}
To define a controller, the dynamics of the system must be first put into state space form:
\[
x = [q,\dot{q}]^T
\]

\[
\frac{d\dot{q_{4}}}{dt} =
\frac{-1}{((I_{yy}-I_{zz}){cos}^2(q_{6})+I_{zz})(2I_{xx}{sin}^2(q_{5}))+(I_{yy}-I_{zz}){cos}^2(q_{5}){cos}^2(q_{6})}
\]

\end{document}

