\documentclass[9pt]{article}
\usepackage{mathtools}
\usepackage{graphicx}
\usepackage{layout}
\setlength{\voffset}{-0.75in}
\setlength{\headsep}{5pt}
\begin{document}

\title{Dynamics Of A Quadcopter}
\author{Daniel Wood}
\maketitle
\section{Introduction}

In recent years the development of smaller and cheaper electronics, specifically inertial measurement systems, has led to a rise in popularity of lightweight unmanned aerial vehicles. Unmanned aerial vehicles come in many configurations, some of the more common are quadrotors, octocopters as well as single engine. This work in particular will focus entirely on the quadrotor configuration. 

The increase in general usage of quadrotor vehicles, has led to the need to autonomously control these vehicles, some applications include automated delivery services, military usage, aerial imagery. In recent years companies have begun implementing tethering features that allow quadrotors to autonomously follow a target. More sophisticated systems utilize autonomous path tracking. This path tracking is realized through a robust control scheme.

The formation of autonomous control consists of a valid dynamical model of the vehicle, and a fast, stabile, robust controller. The dynamics of a quadrotor is not trivial and requires an indepth understanding of advanced dynamics topics. The unsimplified dynamical model inherently leads to a non-linear controller. There are many control algorithms to choose from, some that are currently being utilized are discussed. Picking a control algorithm stems from the goal of making the output stable as well as having a robust controller. A system is said to be stable if it can remain at an equilibrium point under the stress of external disturbances [cite something ehre], whereas robustness is defined as a controller that can operate under parameters that vary from the original design [ cite something here].

There are many researchers dedicating time to improving controller stability through different control schemes such as Multiple Input Multiple Output state variable (MIMO)\cite{FarameeVeeravat2014EotS}, Linear Quadratic Regulator (LQR), Model Predictive Control (MPC), and Sliding Mode Control (SMC). In general this research focuses on nuances of the choosen algorithm with the goal of improvement in disturbance rejection\cite{DenisKotarski2016CDFU} and robustness. Because of the scope of the research, the dynamics model of the quadrotor is generally linearized for simplicity, or assumptions that were made lead to reduced order of the system\cite{FarameeVeeravat2014EotS}. Higher order dynamics that are missed when a model is linearized can contribute to errors in the control scheme. 

The goal of this research is to start with the concrete fundamentals of dynamics to derive a state space model of the quadrotor in 3D space, relative to a fixed global frame. Then create a stable and robust non-linear controller for accurate path tracking. This controller will be based on the Sliding Mode Control (SMC) method, which is a specific type of variable structure system. This method is a valid choice for this dynamical system because it allows the controller to quickly jump to different equilibrium points, and forces state variables towards the equilibrium points, or in the case of SMC, the sliding manifold as it is noted in literature.

Derivation of the equations of motion of the quadrotor requires defining two coordinate systems, the global, or fixed frame, and the body, or inertial frame\cite{FanniMohamed2017AN6Q},\cite{FarameeVeeravat2014EotS},\cite{HaomiaoHuang2009Aaco},\cite{DenisKotarski2016CDFU}. The path that is defined is always given in reference to the global frame, whereas the equations of motion that define our controllable states, is in reference to the body frame. Rotations of the body such as yaw, pitch, roll, are determined through sensors such as accelerometers and IMUs physically placed on the body. While physical position in space is defined by a global positioning system (GPS), which requires knowledge of the body as well as the reference frame.

Sliding mode control is a technique that allows

\bibliography{References}
\bibliographystyle{plain}
\end{document}